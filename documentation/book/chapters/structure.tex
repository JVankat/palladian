\chapter{Toolkit Structure}
\label{sec:toolkitstructure}

Palladian's source code is managed using Git\footnote{\url{http://git-scm.com/}}, which offers several advantages compared to other version control systems like SVN. To get familiar with Git, we recommend looking at one of the several tutorials or cheat sheets which can be found with the search engine of your choice. Git is available for all major platforms. You can either use it via command line, utilize a graphical stand alone frontend like TortoiseGit\footnote{\url{http://code.google.com/p/tortoisegit/}} on Windows or SourceTree \footnote{\url{http://sourcetreeapp.com}} on Mac OS X or install the plugin EGit\footnote{\url{http://eclipse.org/egit/}; if you are using Eclipse 3.6 or newer, you can install EGit using the Eclipse Marketplace.} to integrate Git's functionality directly into Eclipse.

The Palladian Git repository is located at Bitbucket\footnote{\url{http://bitbucket.org/}}. To receive access to the repository, first register an account at Bitbucket, then contact one of the Palladian developers and ask for an activation of your account, stating your username. The URL for the main Palladian repository is as follows:

\begin{verbatim}
https://bitbucket.org/palladian/palladian.git
\end{verbatim}

To obtain a checkout using Git's command line tool, you typically enter:

\begin{verbatim}
git clone https://johndoe@bitbucket.org/palladian/palladian.git
\end{verbatim}

\section{Directories}

The typical directory structure after a successful checkout looks as follow.

\begin{verbatim}
palladian
 |- config
 |- data
    |- evaluation
    |- temp
    |- test
 |- documentation
    |- book
    |- javadoc
    |- slides
 |- exe
 |- src
    |- main
       |- java
       |- resources
    |- test
       |- java
       |- resources
 |- dev
\end{verbatim}

\section{Config Folder}
\label{sec:config.conf}
The config folder contains a template for the default Palladian configuration file and the database schema needed to run some of the components. The files are explained in the following sections.

\subsection{palladian.properties.default}
\label{sec:palladian.properties.default}
The file \texttt{palladian.properties.default} is a template for the main Palladian configuration file. You need to copy this file and rename it to \texttt{palladian.properties}. The file may be located at one of three positions. The toolkit checks them in the following order:
\begin{enumerate}
\item The folder config inside folder on the local file system specified by the environment variable \texttt{\$PALLADIAN\_HOME}.
\item In the root of the classpath the application is run from. This usually is the root of the JAR you are running or the root folder where all \texttt{*.class} files are located.
\item Inside the \texttt{config} folder at the folder you are running your Palladian powered application from.
\end{enumerate}

The configuration file is structured into several sections explained in the following paragraphs. Inside your own copy of \texttt{palladian.properties} you need to adapt these settings to your local requirements.

\paragraph{API keys}
\label{sec:apikeys.conf}
The api keys that are used by the toolkit components are specified here. You may need to apply for API keys at the provider's page.

\begin{verbatim}
api.alchemy.key = 
api.bing.key = 
api.google.key = 
api.google.tranlsate.key = 
api.hakia.key =
api.opencalais.key =
api.yahoo.key = 
api.yahoo_boss.key = 
api.bitly.login = 
api.bitly.key = 
api.mixx.key = 
api.reddit.username = 
api.reddit.password = 

# access to Collecta XMPP and REST API
# see: http://developer.collecta.com/APIindex/
api.collecta.key = 

api.majestic.key = 
api.compete.key = 
\end{verbatim}

\section{Data Folder}
The data folder contains files that are used during runtime of several components.

\subsection{Model Folder}
The models folder contains learned models that can be reused. See section \ref{sec:AccessModelsDataSets} for an explanation on how to access our extensive collection of data sets and models.

\subsection{Temp Folder}
The temp folder is not part of the repository. Some functions may however create this folder and write temporary data. It is not part of the SVN and should not be committed when automatically created.

\subsection{Test Folder}  
The test folder contains data that is used for running jUnit tests. It is not part of the SVN and should not be committed when automatically created.

\section{Documentation Folder}
The documentation folder contains help files to understand the toolkit. This very document is located in the book folder and a Javadoc can be found here too.

\section{Exe Folder}
The exe folder contains all runnable jar files in separate folders including a sample script to run the program and a readme.txt that explains the run options.

\section{Src Folder}
The src folder contains all source files of the toolkit. You may need to put the log4j.properties file here in order to use custom logging settings. Alternatively you include the config folder in the class path.
