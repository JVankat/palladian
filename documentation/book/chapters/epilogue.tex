\chapter{Where to Go from Here?}
If you can't find something that you need, there is a list of similar projects in Section \ref{sec:alternativesToPalladian} that you can scan through. If you still can't find it, research the topic, implement the code and commit it back to Palladian.

\section{Referenced Libraries}
Palladian makes excessive use of third party libraries. We do not intend to re-implement code but rather to built on it and create something superior. Here an incomplete list of libraries the toolkit uses:
\begin{itemize}
\item Apache Commons \cite{apachecommons} for many standard tasks in string and number manipulation and more.
%\item Fathom \cite{fathom} to measure readability of English text.
\item iText \cite{itext} for creating PDF documents.
%\item Jena \cite{jena} for reading and writing ontology files.
%\item jYaml \cite{jyaml} to read and write YAML files.
\item Log4j \cite{log4j} for logging.
\item Lucene \cite{lucene} for indexing and making learned models persistent.
\item NekoHTML \cite{nekohtml} to clean up the HTML of web pages in order to process them correctly.
\item ROME \cite{rome} for parsing RSS and Atom feeds.
\item SimMetrics \cite{simmetrics} to calculate similarities of strings.
\item Weka \cite{hall2009weka} for machine learning.
\end{itemize}

\section{History}
The foundation of Palladian's code came out of the WebKnox project\cite{webknox} that was started in 2008.% Now, components of the Aletheia project\footnote{\url{http://www.aletheia-projekt.de}} and the Effingo project\footnote{\url{http://www.effingo.de}}

The code is in development by students of the Dresden University of Technology. Contributors are:
% Clean up, remove noninvolved contributors
\begin{itemize}
\item Christopher Friedrich
\item Martin Gregor
\item Philipp Katz
\item Klemens Muthmann
\item Silvio Rabe
\item Sandro Reichert
\item Julien Schmehl
\item David Urbansky
\item Robert Willner
\item Martin Werner
\item Martin Wunderwald
\item Stephan Zepezauer
\end{itemize}

%\chapter{References}
\bibliographystyle{abbrv}
\bibliography{references}
